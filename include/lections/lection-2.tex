\chapter{Понятие «открытая система» и проблемы стандартизации}

\begin{enumerate}
    \item Организация взаимодействия между устройствами в сети.
    \item Модель OSI.
    \item Основные функции уровней модели OSI.
    \item Понятие «открытая система».
\end{enumerate}

Тезис о пользе стандартизации, справедливый для всех отраслей, в компьютерных сетях приобретает особое значение.
Суть сети - это соединение разного оборудования, а значит, проблема совместимости для сетей является одной из наиболее острых.
Поэтому все развитие этой отрасли, в конечном счете, отражено в стандартах - любая новая технология только тогда приобретает «законный» статус, когда ее содержание закрепляется в соответствующем стандарте.

В компьютерных сетях идеологической основой стандартизации является многоуровневый подход к разработке средств сетевого взаимодействия.
Именно на основе этого подхода была разработана стандартная семиуровневая модель взаимодействия открытых систем, ставшая своего рода универсальным языком сетевых специалистов.

\section{Многоуровневый подход. Протокол. Интерфейс. Стек протоколов}

Организация взаимодействия между устройствами в сети является сложной задачей, для решения которой используется универсальный прием - декомпозиция, то есть разбиение одной сложной задачи на несколько более простых задач-модулей.

При декомпозиции часто используют многоуровневый подход.
Он заключается в следующем.
Все множество задач-модулей разбивают на уровни.
Уровни образуют иерархию, то есть имеются вышележащие и нижележащие уровни, причем взаимодействие возможно только между соседними (близлежащими) уровнями.

Такая иерархическая декомпозиция задачи предполагает четкое определение функция каждого уровня и интерфейсов между уровнями.
Интерфейс определяет набор функций, которые нижележащий уровень предоставляет вышележащему.
В результате иерархической декомпозиции достигается относительная независимость уровней, а значит, и возможность их легкой замены.

\subsection{Вариант модели взаимодействия двух узлов. Понятие протокола, интерфейса}

Многоуровневое представление средств \textbf{\textit{сетевого взаимодействия}} имеет свою специфику, связанную с тем, что в процессе обмена сообщениями участвуют, \textbf{\textit{как минимум две машины}}, то есть в данном случае \textbf{\textit{необходимо организовать согласованную работу двух «иерархий»}}.

Один из возможных вариантов модели взаимодействия двух узлов на основе многоуровневого подхода показан на рисунке.

С каждой стороны средства взаимодействия представлены четырьмя уровнями.
Процедура взаимодействия этих двух узлов может быть описана в виде набора правил взаимодействия каждой пары соответствующих уровней обеих участвующих сторон.
Формализованные правила, определяющие последовательность и формат сообщений, которыми обмениваются сетевые компоненты, лежащие на одном уровне, но в разных узлах, называются \textbf{\textit{протоколом}}.

Модули, реализующие протоколы соседних уровней и находящиеся в одном узле, также взаимодействуют друг с другом в соответствии с четко определенными правилами и с помощью стандартизованных форматов сообщений.
Эти правила принято называть \textbf{\textit{интерфейсами}}.
Интерфейс определяет набор сервисов, предоставляемый данным уровнем соседнему уровню.
В сущности, протокол и интерфейс выражают одно и то же понятие, но традиционно в сетях за ними закрепили разные области действия: протоколы определяют правила взаимодействия модулей одного уровня в разных узлах, а интерфейсы - модулей соседних уровней в одном узле.

Средства каждого уровня должны отрабатывать, во-первых, свой собственный протокол, а во-вторых, интерфейсы с соседними уровнями.

\subsection{Понятие стека}

Иерархически организованный набор протоколов, достаточный для организации взаимодействия узлов в сети, называется \emph{стеком коммуникационных протоколов}.

Протоколы реализуются не только компьютерами, но и другими сетевыми устройствами - концентраторами, мостами, коммутаторами, маршрутизаторами и т.д.

\section{Модель OSI}

В начале 80-х годов ряд международных организаций по стандартизации - ISO, ITU-T и некоторые другие - разработали модель, которая \textbf{\textit{называется моделью взаимодействия открытых систем (Open System Interconnection, OSI) или моделью OSI}}.
Модель OSI определяет различные уровни взаимодействия систем, дает им стандартные имена и указывает, какие функции должен выполнять каждый уровень.

В модели OSI имеются семь уровней:
\begin{itemize}
    \item прикладной,
    \item представительный,
    \item сеансовый,
    \item транспортный,
    \item сетевой,
    \item канальный;
    \item физический.
\end{itemize}

Каждый уровень имеет дело с одним определенным аспектом взаимодействия сетевых устройств.

\subsection{Принцип обслуживания запроса приложения в модели OSI}

Итак, пусть приложение обращается с запросом к \textbf{\textit{прикладному уровню}}, например к файловой службе.
На основании этого запроса программное обеспечение прикладного уровня формирует сообщение стандартного формата.
Обычное сообщение состоит из заголовка и поля данных.
Заголовок содержит служебную информацию, которую необходимо передать через сеть прикладному уровню машины-адресата, чтобы сообщить ему, какую работу надо выполнить.
В нашем случае заголовок, очевидно, должен содержать информацию о месте нахождения файла и о типе операции, которую необходимо над ним выполнить.

После формирования сообщения прикладной уровень направляет его вниз по стеку \textbf{\textit{представительному уровню}}.
Протокол представительного уровня на основании информации, полученной из заголовка прикладного уровня, выполняет требуемые действия и добавляет к сообщению собственную служебную информацию - заголовок представительного уровня, в котором содержатся указания для протокола представительного уровня машины-адресата.
Полученное в результате сообщение передается вниз \textbf{\textit{сеансовому уровню}}, который в свою очередь добавляет свой заголовок, и \textbf{\textit{т.д.}}
Наконец, сообщение достигает нижнего, физического уровня, который собственно и передает его по линиям связи машине-адресату.

Когда сообщение по сети поступает на машину-адресат, оно принимается ее физическим уровнем и последовательно перемещается вверх с уровня на уровень.
Каждый уровень анализирует и обрабатывает заголовок своего уровня, выполняя соответствующие данному уровню функции, а затем удаляет этот заголовок и передает сообщение вышележащему уровню.

\subsubsection{Понятие протокольного блока данных}

Наряду с термином сообщение (message) существуют и другие термины, применяемые сетевыми специалистами для обозначения единиц данных в процедурах обмена.
В стандартах ISO для обозначения единиц данных, с которыми имеют дело протоколы разных уровней, используется общее название - протокольный блок данных (Protocol Data Unit, PDU).
Для обозначения блоков данных определенных уровней часто используются специальные названия: кадр (frame), пакет (packet), дейтаграмма (datagram), сегмент (segment).

\section{Уровни модели OSI}

В модели OSI имеются семь уровней:
\begin{itemize}
    \item прикладной,
    \item представительный,
    \item сеансовый,
    \item транспортный,
    \item сетевой,
    \item канальный;
    \item физический.
\end{itemize}

\subsection{Физический уровень}

\textbf{\textit{Физический уровень (Physical layer)}} имеет дело с передачей битов по физическим каналам связи, таким, например, как коаксиальный кабель, витая пара, оптоволоконный кабель или цифровой территориальный канал.
К этому уровню имеют отношение характеристики физических сред передачи данных, такие как полоса пропускания, помехозащищенность, волновое сопротивление и другие.
На этом же уровне определяются характеристики электрических сигналов, передающих дискретную информацию, например, крутизна фронтов импульсов, уровни напряжения или тока передаваемого сигнала, тип кодирования, скорость передачи сигналов.
Кроме этого, здесь стандартизуются типы разъемов и назначение каждого контакта.

Функции физического уровня реализуются во всех устройствах, подключенных к сети.
Со стороны компьютера функции физического уровня выполняются сетевым адаптером или последовательным портом.
Примером протокола физического уровня может служить спецификация 10Ваsе-Т технологии Ethernet, которая определяет в качестве используемого кабеля неэкранированную витую пару категории 3 с волновым сопротивлением 100 Ом, разъем RJ-45, максимальную длину физического сегмента 100 метров, манчестерский код для представления данных в кабеле и т.д.

\subsection{Канальный уровень}

На физическом уровне просто пересылаются биты.
При этом не учитывается, что физическая среда передачи может быть занята.
Поэтому одной из задач \textbf{\textit{канального уровня (Data Link layer)}} является \textbf{\textit{проверка доступности среды передачи}}.
Другой задачей канального уровня является \textbf{\textit{реализация механизмов обнаружения и коррекции ошибок}}.
Для этого на канальном уровне биты группируются в наборы, называемые \emph{кадрами (frames)}.
Канальный уровень обеспечивает корректность передачи каждого кадра, помещая специальную последовательность бит в начало и конец каждого кадра для его выделения, а также вычисляет контрольную сумму, обрабатывая все байты кадра определенным способом и добавляя контрольную сумму к кадру.
Когда кадр приходит по сети, получатель снова вычисляет контрольную сумму полученных данных и сравнивает результат с контрольной суммой из кадра.
Если они совпадают, кадр считается правильным и принимается.
Если же контрольные суммы не совпадают, то фиксируется ошибка.
Канальный уровень может не только обнаруживать ошибки, но и исправлять их за счет повторной передачи поврежденных кадров, хотя функция исправления ошибок не является обязательной для канального уровня.

В протоколах канального уровня, используемых в локальных сетях, заложена определенная структура связей между компьютерами и способы их адресации.
Хотя канальный уровень и обеспечивает доставку кадра между любыми двумя узлами локальной сети, он это делает только в сети с той топологией связей, для которой был разработан.
Примерами протоколов канального уровня являются протоколы Ethernet, Token Ring, FDDI, l00VG-AnyLAN.

В локальных сетях протоколы канального уровня используются компьютерами, мостами, коммутаторами и маршрутизаторами.
В компьютерах функции канального уровня реализуются совместными усилиями сетевых адаптеров и их драйверов.

В глобальных сетях, которые редко обладают регулярной топологией, канальный уровень часто обеспечивает обмен сообщениями только между двумя соседними компьютерами, соединенными индивидуальной линией связи.

В целом канальный уровень представляет собой весьма мощный и законченный набор функций по пересылке сообщений между узлами сети.
В некоторых случаях протоколы канального уровня оказываются самодостаточными транспортными средствами и могут допускать работу поверх них непосредственно протоколов прикладного уровня или приложений, без привлечения средств сетевого и транспортного уровней.
Тем не менее, для обеспечения качественной транспортировки сообщений в сетях любых топологий и технологий функций канального уровня оказывается недостаточно, поэтому в модели OSI решение этой задачи возлагается на два следующих уровня - сетевой и транспортный.

\subsection{Сетевой уровень}

\textbf{\textit{Сетевой уровень (Network layer)}} служит для образования единой транспортной системы, объединяющей несколько сетей, причем эти сети могут использовать совершенно различные принципы передачи сообщений между конечными узлами и обладать произвольной структурой связей.
Начнем рассмотрение функций сетевого уровня на примере объединения локальных сетей.

Протоколы канального уровня локальных сетей обеспечивают доставку данных между любыми узлами только в сети с соответствующей типовой топологией, например топологией иерархической звезды.
Это очень жесткое ограничение, которое не позволяет строить сети с развитой структурой, например, сети, объединяющие несколько сетей предприятия в единую сеть, или высоконадежные сети, в которых существуют избыточные связи между узлами.
Можно было бы усложнять протоколы канального уровня для поддержания петлевидных избыточных связей, но принцип разделения обязанностей между уровнями приводит к другому решению.
Чтобы с одной стороны сохранить простоту процедур передачи данных для типовых топологий, а с другой допустить использование произвольных топологий, вводится дополнительный сетевой уровень.

На сетевом уровне сам термин \emph{сеть} наделяют специфическим значением.
В данном случае под сетью понимается совокупность компьютеров, соединенных между собой в соответствии с одной из стандартных типовых топологий и использующих для передачи данных один из протоколов канального уровня, определенный для этой топологии.

Внутри сети доставка данных обеспечивается соответствующим канальным уровнем, а вот \textbf{\textit{доставкой данных между сетями занимается сетевой уровень}}, который и поддерживает возможность правильного выбора маршрута передачи сообщения даже в том случае, когда структура связей между составляющими сетями имеет характер, отличный от принятого в протоколах канального уровня.

Сети соединяются между собой специальными устройствами, называемыми маршрутизаторами.
Маршрутизатор - это устройство, которое собирает информацию о топологии межсетевых соединений и на ее основании пересылает пакеты сетевого уровня в сеть назначения.

Проблема выбора наилучшего пути называется маршрутизацией, и ее решение является одной из главных задач сетевого уровня.

В общем случае функции сетевого уровня шире, чем функции передачи сообщений по связям с нестандартной структурой, которые мы рассмотрели на примере объединения локальных сетей.
Сетевой уровень решает также задачи согласования разных технологий, упрощения адресации в крупных сетях и создания надежных и гибких барьеров на пути нежелательного трафика между сетями.

Сообщения сетевого уровня принято называть \textbf{\textit{пакетами (packets)}}.
При организации доставки пакетов на сетевом уровне используется понятие «номер сети».
В этом случае адрес получателя состоит из старшей части - номера сети и младшей - номера узла в этой сети.
Все узлы одной сети должны иметь одну и ту же старшую часть адреса, поэтому термину «сеть» на сетевом уровне можно дать и другое, более формальное определение: сеть - это совокупность узлов, сетевой адрес которых содержит один и тот же номер сети.

На сетевом уровне определяются два вида протоколов.
Первый вид - \textbf{\textit{сетевые протоколы (routed protocols)}} - реализуют продвижение пакетов через сеть.
Именно эти протоколы обычно имеют в виду, когда говорят о протоколах сетевого уровня.
Однако часто к сетевому уровню относят и другой вид протоколов, называемых протоколами обмена маршрутной информацией или просто \textbf{\textit{протоколами маршрутизации (routing protocols)}}.
С помощью этих протоколов маршрутизаторы собирают информацию о топологии межсетевых соединений.

На сетевом уровне работают протоколы еще одного типа, которые отвечают за отображение адреса узла, используемого на сетевом уровне, в локальный адрес сети.
Такие протоколы часто называют \textbf{\textit{протоколами разрешения адресов (Address Resolution Protocol, ARP)}}.
Иногда их относят не к сетевому уровню, а к канальному, хотя тонкости классификации не изменяют их сути.

Примерами протоколов сетевого уровня являются протокол межсетевого взаимодействия IP стека TCP/IP и протокол межсетевого обмена пакетами IPX стека Novell.

\subsection{Транспортный уровень}

На пути от отправителя к получателю пакеты могут быть искажены или утеряны.
Хотя некоторые приложения имеют собственные средства обработки ошибок, существуют и такие, которые предпочитают сразу иметь дело с надежным соединением.
\textbf{\textit{Транспортный уровень (Transport layer)}} обеспечивает приложениям или верхним уровням стека - прикладному и сеансовому - \textbf{\textit{передачу данных с той степенью надежности, которая им требуется}}.
Модель OSI определяет пять классов сервиса, предоставляемых транспортным уровнем.
Эти виды сервиса отличаются качеством предоставляемых услуг, срочностью, возможностью восстановления прерванной связи, наличием средств мультиплексирования нескольких соединений между различными прикладными протоколами через общий транспортный протокол, а главное - способностью к обнаружению и исправлению ошибок передачи, таких как искажение, потеря и дублирование пакетов.

Выбор класса сервиса транспортного уровня определяется, с одной стороны, тем, в какой степени задача обеспечения надежности решается самими приложениями и протоколами более высоких, чем транспортный, уровней, а с другой стороны, этот выбор зависит от того, насколько надежной является система транспортировки данных в сети, обеспечиваемая уровнями, расположенными ниже транспортного - сетевым, канальным и физическим.
Так, например, если качество каналов  связи очень высокое и вероятность возникновения ошибок, не обнаруженных протоколами более низких уровней, невелика, то разумно воспользоваться одним из облегченных сервисов транспортного уровня, не обремененных многочисленными проверками, квитированием и другими приемами повышения надежности.
Если же транспортные средства нижних уровней изначально очень ненадежны, то целесообразно обратиться к наиболее развитому сервису транспортного уровня, который работает, используя максимум средств для обнаружения и устранения ошибок, - с помощью, предварительного установления логического соединения, контроля доставки сообщений по контрольным суммам и циклической нумерации пакетов, установления тайм-аутов доставки и т.п.

Как правило, все протоколы, начиная с транспортного уровня и выше, реализуются программными средствами конечных узлов сети - компонентами их сетевых операционных систем.
В качестве примера транспортных протоколов можно привести протоколы TCP и UDP стека TCP/IP.

Протоколы нижних четырех уровней обобщенно называют сетевым транспортом или транспортной подсистемой, так как они полностью решают задачу транспортировки сообщений с заданным уровнем качества в составных сетях с произвольной топологией и различными технологиями.
Остальные три верхних уровня решают задачи предоставления прикладных сервисов на основании имеющейся транспортной подсистемы.

\subsection{Сеансовый уровень}

\textbf{\textit{Сеансовый уровень (Session layer)}} обеспечивает управление диалогом: фиксирует, какая из сторон является активной в настоящий момент, предоставляет средства синхронизации.
Последние позволяют вставлять контрольные точки в длинные передачи, чтобы в случае отказа можно было вернуться назад к последней контрольной точке, а не начинать все с начала.
На практике немногие приложения используют сеансовый уровень, и он редко реализуется в виде отдельных протоколов, хотя функции этого уровня часто объединяют с функциями прикладного уровня и реализуют в одном протоколе.

\subsection{Представительный уровень}

\textbf{\textit{Представительный уровень (Presentation layer)}} имеет дело с формой представления передаваемой по сети информации, не меняя при этом ее содержания.
За счет уровня представления информация, передаваемая прикладным уровнем одной системы, всегда понятна прикладному уровню другой системы.
С помощью средств данного уровня протоколы прикладных уровней могут преодолеть синтаксические различия в представлении данных или же различия в кодах символов, например кодов ASCII и ERBDIC.
На этом уровне, может выполняться шифрование и дешифрование данных, благодаря которому секретность обмена данными обеспечивается сразу для всех прикладных служб.
Примером такого протокола является протокол Secure Socket Layer (SSL), который обеспечивает секретный обмен сообщениями для протоколов прикладного уровня стека TCP/IP.

\subsection{Прикладной уровень}

\textbf{\textit{Прикладной уровень (Application layer)}} - это в действительности просто набор разнообразных протоколов, с помощью которых пользователи сети получают доступ к разделяемым ресурсам, таким как файлы, принтеры или гипертекстовые Web-страницы, а также организуют свою совместную работу, например, с помощью протокола электронной почты.
Единица данных, которой оперирует прикладной уровень, обычно называется \emph{сообщением (message)}.

Существует очень большое разнообразие служб прикладного уровня.
Приведем в качестве примера хотя бы несколько наиболее распространенных реализации файловых служб: NCP в операционной системе Novell NetWare, SMB в Microsoft Windows NT, NFS, FTP и TFTP, входящие в стек TCP/IP.

\subsection{Сетезависимые и сетенезависимые уровни}

Функции всех уровней модели OSI могут быть отнесены к одной из двух групп: либо к функциям, зависящим от конкретной технической реализации сети, либо к функциям, ориентированным на работу с приложениями.

Три нижних уровня - физический, канальный и сетевой - являются сетезави-симыми, то есть протоколы этих уровней тесно связаны с технической реализацией сети и используемым коммуникационным оборудованием.
Например, переход на оборудование FDDI означает полную смену протоколов физического и канального уровней во всех узлах сети.

Три верхних уровня - прикладной, представительный и сеансовый - ориентированы на приложения и мало зависят от технических особенностей построения сети.
На протоколы этих уровней не влияют какие бы то ни было изменения в топологии сети, замена оборудования или переход на другую сетевую технологию.
Так, переход от Ethernet на высокоскоростную технологию l00VG-AnyLAN не потребует никаких изменений в программных средствах реализующих функции прикладного, представительного и сеансового уровней.

Транспортный уровень является промежуточным, он скрывает все детали функционирования нижних уровней от верхних.
Это позволяет разрабатывать приложения, не зависящие от технических средств непосредственной транспортировки сообщений.

\section{Понятие «открытая система»}

Модель OSI, как это следует из ее названия (Open System Interconnectlon), описывает взаимосвязи открытых систем.
Что же такое открытая система?

В широком смысле \emph{открытой системой} может быть названа любая система (компьютер, вычислительная сеть, ОС, программный пакет, другие аппаратные и программные продукты), которая построена в соответствии с открытыми спецификациями, под которыми понимаются опубликованные, общедоступные спецификации, соответствующие стандартам и принятые в результате достижения согласия после всестороннего обсуждения всеми заинтересованными сторонами.

Для реальных систем полная открытость является недостижимым идеалом.
Как правило, даже в системах, называемых открытыми, этому определению соответствуют лишь некоторые части, поддерживающие внешние интерфейсы.
Чем больше открытых спецификаций использовано при разработке системы, тем более открытой она является.

\subsection{Открытость модели OSI}

Модель OSI касается только одного аспекта открытости, а именно открытости средств взаимодействия устройств, связанных в вычислительную сеть.
Здесь под открытой системой понимается сетевое устройство, готовое взаимодействовать с другими сетевыми устройствами с использованием стандартных правил, определяющих формат, содержание и значение принимаемых и отправляемых сообщений.

Если две сети построены с соблюдением принципов открытости, то это дает следующие преимущества:
\begin{itemize}
    \item возможность построения сети из аппаратных и программных средств различных производителей, придерживающихся одного и того же стандарта;
    \item возможность безболезненной замены отдельных компонентов сети другими, более совершенными, что позволяет сети развиваться с минимальными затратами;
    \item возможность легкого сопряжения одной сети с другой;
    \item простота освоения и обслуживания сети.
\end{itemize}
